\section{Pencils}

TJ IOI Inc.\@ has reached a net worth of a million dollars! Kevin, the CEO, is very happy of this achievement, and asks Alex to write a report on the financial standing of the company, first thing tomorrow.  However, Kevin decides that instead of typing the report, he will make Alex hand write it, in order to build character.
\blank
That night, Alex has gathered $ N $ ($ 1 \leq N \leq 100,000 $) pencils in front of him. However, his pencil bag only has room for $ K $ pencils ($ 1 \leq K \leq N $). Each of his pencils has an integer length in the range of $ 1 $ to $ 1,000 $, inclusive. Because Kevin expects an extravagant and exhaustively detailed report, Alex will have to write a lot if he wants to please Kevin, so Alex wants to choose the longest $ K $ pencils to place into his pencil bag.  Please help Alex determine the sum of the lengths of those pencils.
\blank
\textbf{SHORT NAME:} \verb|pencils|
\blank
\textbf{INPUT FORMAT:}\\
The first line of input contains two integers $ N $ and $ K $. The next $ N $ lines describe the length of the pencils.
\blank
\textbf{OUTPUT FORMAT:}\\
Output a single integer, the sum of the $ K $ longest pencils.
\blank
\textbf{SAMPLE INPUT:}
\begin{verbatim}
7 3
1 
4 
5 
3 
8 
14 
2
\end{verbatim}
\textbf{SAMPLE OUTPUT:}
\begin{verbatim}
27
\end{verbatim}
