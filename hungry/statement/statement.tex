\documentclass{article}
\usepackage[utf8]{inputenc}
\usepackage[margin=1in]{geometry}
\usepackage{titling}

\renewcommand\maketitlehooka{\null\mbox{}\vfill}
\renewcommand\maketitlehookd{\vfill\null}

\begin{document}

\newcommand{\blank}{\vskip 3mm}
\setlength\parindent{0pt}
\renewcommand\thesection{\Alph{section}}

\setcounter{section}{8}
\section{Hungry Hungry Larrys}

Devon stands at the top left corner of an $ N \times N $ grid ($ 1 \leq N \leq 4000 $), and can move either to the right or down, and would like to reach the bottom right corner.  However, a number of Larrys reside on each cell of the grid.  Every time Devon moves to a new cell, each Larry on the grid cell that Devon moves to will reach into Devon’s wallet and take one dollar.  Thankfully, Devon’s wallet has an infinite amount of money, but he still would like to lose as little money as possible.  It is guaranteed that there are no Larrys on the cell that Devon begins on.  Determine the least amount of money that Devon must lose to the hungry hungry Larrys along the way, in order to reach his destination cell.
\blank
\textbf{INPUT FORMAT:}\\
The first line of input contains the integer $ N $, the size of the grid.  The next $ N $ lines each contain $ N $ integers, and together describe the number of Larrys at each location on the grid.
\blank
\textbf{OUTPUT FORMAT:}\\
Output a single integer, the least amount of money that Devon must lose in order to get to his final cell.
\blank
\textbf{SAMPLE INPUT:}
\begin{verbatim}
4
3 2 5 1
2 9 3 0
4 6 1 2
8 2 2 6
\end{verbatim}
\textbf{SAMPLE OUTPUT:}
\begin{verbatim}
19
\end{verbatim}
\newpage
\end{document}
